\documentclass{article}
\usepackage[pagebackref,letterpaper=true,colorlinks=true,pdfpagemode=none,urlcolor=blue,linkcolor=blue,citecolor=blue,pdfstartview=FitH]{hyperref}

\usepackage{amsmath,amsfonts}
\usepackage{graphicx}
\usepackage{color}


\setlength{\oddsidemargin}{0pt}
\setlength{\evensidemargin}{0pt}
\setlength{\textwidth}{6.0in}
\setlength{\topmargin}{0in}
\setlength{\textheight}{8.5in}


\setlength{\parindent}{0in}
\setlength{\parskip}{5px}

%\input{macrosblog}

%%%%%%%%% For wordpress conversion

\def\more{}

\newif\ifblog
\newif\iftex
\blogfalse
\textrue


\usepackage{ulem}
\def\em{\it}
\def\emph#1{\textit{#1}}

\def\image#1#2#3{\begin{center}\includegraphics[#1pt]{#3}\end{center}}

\let\hrefnosnap=\href

\newenvironment{btabular}[1]{\begin{tabular} {#1}}{\end{tabular}}

\newenvironment{red}{\color{red}}{}
\newenvironment{green}{\color{green}}{}
\newenvironment{blue}{\color{blue}}{}

%%%%%%%%% Typesetting shortcuts

\def\B{\{0,1\}}
\def\xor{\oplus}

\def\P{{\mathbb P}}
\def\E{{\mathbb E}}
\def\var{{\bf Var}}

\def\N{{\mathbb N}}
\def\Z{{\mathbb Z}}
\def\R{{\mathbb R}}
\def\C{{\mathbb C}}
\def\Q{{\mathbb Q}}
\def\eps{{\epsilon}}

\def\bz{{\bf z}}

\def\true{{\tt true}}
\def\false{{\tt false}}

%%%%%%%%% Theorems and proofs

\newtheorem{exercise}{Exercise}
\newtheorem{theorem}{Theorem}
\newtheorem{lemma}[theorem]{Lemma}
\newtheorem{definition}[theorem]{Definition}
\newtheorem{corollary}[theorem]{Corollary}
\newtheorem{proposition}[theorem]{Proposition}
\newtheorem{example}{Example}
\newtheorem{remark}[theorem]{Remark}
\newenvironment{proof}{\noindent {\sc Proof:}}{$\Box$} %\medskip} 
%%%%%%%%% I added
\newtheorem{assumption}{Assumption}
%%%%%%%%

\begin{document}

\begin{center}
 %MA573 - Final Project
\end{center}

%\item 

\section{CRR model}

A binomial tree model is given as below:
\begin{itemize}
 \item Let $X$ be a random walk with transition probability
 $$p(X_{n} = (j+1)\Delta x | X_{n-1} = j \Delta x) = p, 
 \quad 
 p(X_{n} = (j-1)\Delta x | X_{n-1} = j \Delta x) = 1-p.
 $$
 \end{itemize}

A CRR$(N; s, r, \sigma, T)$ model assumes Stock price $S$ satisfies
with the following additional conditions:
\begin{itemize}
 \item $S$ starts from $s$;
 \item $X_{t} = \ln S_{t}$ follows the random walk with transition probability 
to upper leg probability 
$$\hat p = \frac{e^{r\Delta t} - e^{-\sigma \sqrt{\Delta t}}}
{e^{\sigma \sqrt{\Delta t}} - e^{-\sigma \sqrt{\Delta t}}}$$
and down leg probability $1- \hat p$.
\end{itemize}

[Q.] 
\begin{itemize}
\item Is CRR arbitrage free model?
\item What is the limit behavior of $X$ as $N\to \infty$?
\end{itemize}


\section{Computation of CRR vanilla options}




\subsection{Vanilla call/put option computing}

A pricing engine for vanilla call/put can be designed by either Backward iteration or Monte-carlo method.
One shall compare two methods by using the following example.

We consider an option with 
\begin{center}
 type = call, maturity $T = 1$ and strike $K = 100$.
\end{center}
Its underlying stock has 
\begin{center}
 spot price $S_{0} = 100$ and volatility $\sigma = 20\%$.
\end{center}
Moreover, the interest rate is $5\%$.

You are expected to try the following tasks:
\begin{enumerate}
 \item Compute BSM call value.
 \item Compute the value of CRR(N = 2000) and observe if it is sufficiently close to BSM value?
 \item Next, we want to demonstrate the convergence 
\begin{center}
 CRR(N) $\to$ BSM, as $N\to \infty$.
\end{center} 
To do that, we take the number of steps 
$N = 10 + 20k$, where $k$ ranges over all integers in $\{0, 1, \ldots, 49\}$.

{\bf pseudocode} 
\begin{itemize}
\item plot a graph of the mapping, for $k \in range(50)$
$$k \mapsto CRR(10 + 20k).$$ 
\item on the same figure, plot a horizontal line with the level of BSM value.
\item observe if the CRR curve actually converging to the BSM line?
\end{itemize}

\item Repeat the above convergence demonstration by replacing the number of steps by
$$N = 10 + 25 k.$$
Observe the CRR curve converging to BSM line in the same manner?

\end{enumerate}


\subsection{Extension}
One can try to extend the function of the pricing engine by Monte-Carlo method to some other options:
\begin{itemize}
\item Digital option
\item American option
\item Barrier option
\item Option with default risk
\item Any useful exotic option pricing
\end{itemize}




Some other investigations are possible if the pricing engine is reliable, for instance
\begin{itemize}
\item How does option price change as volatility is getting bigger?
What do you think of the price if $\sigma\to 0$ or $\sigma\to \infty$?
\item How sensitively does the price change as the spot price is getting bigger? More sensitivity questions can be referred to Greeks.
\item Any other useful features for discovery?
\end{itemize}




\section{Further investigations}

\begin{itemize}
\item Can one improve pricing on Digital option and Barrier option? See importance sampling/variance reduction, etc.
\item American option is indeed MDP and one can try one can try any of reinforcement learnings. Probably one can start with Q-learning.
\item Try Q-learning to American option with barrier/default risk.
\item Try deep BSDE to American option with barrier/default risk. 
\end{itemize}














%\bibliographystyle{plain}
%\bibliographystyle{plainnat}
%\bibliographystyle{apalike}
%\bibliography{/Users/songqsh/Dropbox/R/refs}

\end{document}
